\indent For spatial data visualization, we approach two problems and provide solutions: heat map resolution selection, and heat map confidence interval presentation. Analysts often present spatial data in gridded heat maps, at some chosen resolution. However, many data types vary in density across the domain. We develop varyiable-resolution heat maps to visually accommodate this changing density, and an R package, {\bf varyres}, to implement it. Further, heat map confidence intervals typically consist of two heat maps, one for each confidence interval bound. We develope an interactive heat map confidence interval that changes dynamically as a user moves through the interval surfaces; and an R package, {\bf mapapp}, to implement it.

For spatial data analysis, Bayesian hierarchical models work well for accommodating complex spatial correlation structures. However, with big N spatial data we face a computational bottleneck on the order of $n^{3}$. We delineate, use, and assess three approaches to addressing the "big N" problem with our spatial baseball strike zone data.



