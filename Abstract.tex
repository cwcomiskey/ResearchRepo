\indent For spatial data visualization, we approach two problems and provide solutions: heat map resolution selection, and heat map confidence interval presentation. Analysts often present spatial data in gridded heat maps, at some chosen resolution. However, many data types vary in density across the domain. We propose varying-resolution heat maps to visually accommodate this changing density. Further, heat map confidence intervals (CI) typically consist of two heat maps, one for each CI bound. We propose an interactive heat map CI that changes dynamically as a user moves through the CI.

For spatial data analysis, Bayesian hierarchical models work well for accommodating complex spatial correlation structures. However, with {\it big} spatial data we face a computational bottleneck on the order of $n^{3}$. We discuss three approaches to confronting the "big N" problem with our spatial baseball strike zone data, and present preliminary assessments.



