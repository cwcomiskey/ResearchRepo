
\section{Introduction}

As the old adage says, a picture is worth a thousand words. Statisticians, and scientists in general, rely on graphical displays---pictures, essentially---to communicate information, analysis, and results. We deploy graphical displays to efficiently communicate data set information content to other statisticians, and, perhaps more importantly, to non-statisticians. As technology generates more data, and statistical analysis becomes more widespread, graphical displays become more important; therefore, our graphic-making abilities must improve. The R data visualization package \verb|ggplot2| stands out as the single most downloaded R package, which highlights the importance of graphical displays and the continued need for graphical innovations \citep{rdoc}.

In this chapter we focus on one type of graphical display: heat maps. We argue uniform resolution throughout the heat map domain, the current resolution best practice, inadequately accomodates spatially varying data density. We offer a solution that addresses this shortcoming.

\section{Bernoulli Swings}

% Note the variable \verb|des|, short for `description,' that describes pitch outcomes. In this study, we use swing outcomes described in \verb|des| to define a Bernoulli random variable $S$ (Section 3.1) that equals one for a hit, and equals zero for {\it any} swing that does not result in a hit. 

Baseball centers around a series of contests between the hitter and the pitcher, comprised of pitches the hitter can swing at, or choose not to swing at. We treat every swing as a trial, and success or failure should be evaluated independently from the count in the at bat at the time of the trial. Accordingly, we consider every swing a trial in this study.  This differs from other studies that include only at bat ending pitches \citep{Cross2015}, \citep{Baumer2010}, \citep{Fast2011}. These studies {\it exclude} from analysis swinging strikes that do not end at bats, and foul balls; but {\it include} pitches that end at bats, even if the hitter did not swing. We consider these latter events a possible mistake in the hitter's decision making, not a failed swing attempt. Based on these rationales, in this work we consider every swing a trial, with some probability of success. 

Accordingly, we define success as swings where the variable \verb|des|, short for description, equals \verb|in play, no out|, and failure as swings where \verb|des| equals \verb|Foul|, \verb|Foul (Runner Going)|, \verb|Foul Tip|, \verb|In play out(s)|, \verb|Swinging Strike|, or \\ \verb|Swinging Strike (Blocked)|. With this interpretation of the data, next we explain the setup and interpretation of a baseball strike zone heat map.

\section{The Strike Zone and Empirical Heat Maps}

Empirical baseball strike zone heat maps cover the two-dimensional, vertical face of the strike zone with a grid, containing empirical success probabilities ($\hat{p}$, defined below) in each grid box.  We start with a data set containing PITCHf/x\textsuperscript{\textregistered} data on 1,932 right-handed hitters, taking 1,582,581 swings between 2008 and 2015.  Let $b = 1, \dots, 627$ index grid boxes. Let $i = 1, \dots, 1,582,581$ index swings, and define $n_{b} = \displaystyle\sum_{i} \text{I}_{\{i \in b \}}$ as the total number of swings in box $b$.

Define Bernoulli random variable, $S_{i}$, to equal one for swing success and zero for swing failure, and define $\hat{p}_{b} = \frac{1}{n_{b}} \displaystyle\sum_{i} S_{i} \cdot \text{I}_{\{i \in b \}}$ as the empirical box $b$ success probability. Figure 1 displays the resulting empirical heat map for this data. The graphic maps the empirical hitter success probability, $\hat{p}_{b}$, to a color on a spectrum, for pitches that passed through the space represented by that grid box.
  \begin{figure}[H]
	\centering
	\includegraphics[scale=.35]{Images/SZandMothership.jpg} 
  \caption{The setup, and the heat map. The setup in the image on the left demonstrates the proper perspective for understanding the heat map on the right. This heat map grids the hitting zone with approximately 3/4 inch by 3/4 inch boxes. The dashed blue line on the right outlines, roughly, the yellow box on the left. Each grid box color represents the empirical success probability ($\hat{p}_{b}$) of hitter swings at pitches passing through that box.  The data consists of 1,932 right handed hitters, swinging at 1,582,581 pitches between 2008 and 2015.}
	\end{figure} 
While not sophisticated statistically, the graphic efficiently conveys empirical spatial success probabilities, by mapping the statistic $\hat{p}_{b}$ to the color spectrum. However, note that the statistician---not the data---determines the map's resolution.

\section{Resolution} % ============== =============

A heat map's creator {\it chooses} a resolution, thereby determining a uniform grid box size for the heat map prior to audience viewing. This decision carries consequences; resolution selection influences heat map quality and appearance markedly, in much the same way bin width selection influences histograms. Also, note that heat map resolution fails to convey varying spatial data density through the domain, or the strike zone in our case. Thus, the viewer gets no spatial density information, and therefore no understanding of spatially varying estimate sample sizes, variances, etc. As a rule, heat maps do not communicate this information.

The heat map in Figure 1.1 divided the strike zone into relatively small boxes, because the data supported it. By ``supported it'' we mean the small, spatially specific boxes retained a sample size large enough to prevent unacceptable $\hat{p}_{b}$ variance inflation. Defining ``acceptable'' variance depends on context and objectives. For example, a pitching coach might want estimates accurate to within 10 batting average points, 95\% of the time. This margin of error, 0.02, requires a sample size of 32 when $p_{b} = 0.09$.\footnote{Note that the variance depends on the mean for a Bernoulli random variable.} In contrast to the data set used for Figure 1.1, individual hitter data sets vary dramatically in size; ranging from a single swing to over 10,000 swings. 

To reiterate, the choice of resolution can dramatically affect heat map quality, appearance, and the usefulness of the depicted parameter estimates. The decision usually depends on the size and nature of the data set in question, and its spatial dispersion through the domain. In the next section we explore this decision in detail.

\section{Resolution Selection}

In this section, we use batter 425509, a veteran player named Jhonny Peralta, to explore resolution selection and its implications. Our data set includes 9,177 Peralta swings. Peralta's swing data yields the heat map in Figure 1.2, which divides the hitting zone into 16 equally sized boxes. Each box maps $\hat{p}_{b}$ to a color, with box sample sizes, $n_{b}$, printed on box centers. We will use box sample sizes to reference boxes. For example, we will call the box in the lower-left ``box 22.'' (**NOTE: MAKE LABELS BIGGER)
        \begin{figure}[H]
      	\centering
      	\includegraphics[scale=.35]{Images/Chapter4x4.pdf} 
      	\caption{This four by four heat map shows the empirical batting average of Johnny Peralta, for pitches passing through the space represented by each of 16 square regions of the hitting zone. Each box maps $\hat{p}_{b}$ to a color, with box sample sizes, $n_{b}$, printed on box centers.}
      	\end{figure} 
In just a single heat map, as a single resolution, interpretable patterns already appear. For example, why did Peralta see so many fewer pitches in box 22 than in the four center boxes?  We interpret this sample size contrast in baseball terms as an example. 

Three primary factors influence pitch location: pitcher game theoretic strategy, pitch location margin of error (distance by which a pitch misses its intended target), and game state. Game theoretic strategy concerns the pitcher's knowledge of the hitter's strengths and weaknesses, and the hitter's reciprocal knowledge. Margin of error concerns the pitcher's tendency to miss his intended target by some amount. The game state includes the count (number of balls and strikes), the number of outs, and baserunner presence/absence. Two example game state pressures include the increased penalty for throwing a pitch outside the strike zone on a three ball count (the runner gets on base at four balls); and the increased penalty for a hit with a runner in ``scoring position.'' Our data includes pitches across all game states, so we will not rely on them for interpretation. However, we can use game theory and margin of error rationale. 

Of all low or inside (left-hand side) pitch locations, Peralta enjoyed the most success low and inside, in box 22. The orange color of box 22, compares favorably to the otherwise green, yellow, and blue low or inside pitches. Therefore, we speculate that pitchers simply avoided this location when pitching low or inside. On the other hand, Peralta enjoyed as much or more success in middle of the strike zone, boxes 1694, 1642, 1274, and 1694; and saw and swung at far more pitches there. Why did pitchers not avoid those locations too? They probably tried! However, game state pressures frequently compel pitchers to throw a strike, and the middle of the strike zone offers the pitcher his best chance; missing the target in the middle of the strike zone will, more often than any other location, still result in a strike. In addition, Peralta, like almost all hitters, strives to swing almost every time the pitcher accidentally throws the easiest pitch to hit.

Getting back to resolution considerations, the four by four resolution suffices for box 22; further subdivision might yield trivially small subdivided-box sample sizes. On the other hand, the center boxes, all with sample sizes above 1200, can and should provide more location specific $\hat{p}$ estimates. This motivates finer resolution in that region of space, but at the expense of box 22. With this trade-off in mind, we subdivide all boxes further. 

For simplicity we divide each box into four equally sized sub-boxes. Note, however, that subdivision algorithms could take countless forms. Figure 3 shows the eight by eight result.
        \begin{figure}[H]
      	\centering
      	\includegraphics[scale=.25]{Images/Chapter16x16.pdf} 
      	\caption{This eight by eight heat map shows the empirical hitting success probability of Johnny Peralta, for pitches passing through the space represented by each of 64 square regions of the hitting zone. Each box maps $\hat{p}_{b}$ to a color, with box sample sizes, $n_{b}$, printed on box centers. A grey box indicates no pitches passed through that box. Notice that this resolution imparts additional information in the center of the hitting zone, some box sample sizes toward the margin have dropped uninformatively low.}
      	\end{figure} 

The centermost 16 boxes still contain sample sizes sufficient to support low variance $p$ estimates; the minimum of these boxes contains 184 swings. Globally, 24 boxes consist of over 150 swings; and 15 boxes still include more than than 250 swings. These boxes could support further subdivision. On the other hand, many boxes, especially edge boxes, now contain sample sizes generally insufficient to support low variance estimates of $p_{b}$. Twenty-nine boxes contain fewer than 50 swings, and 17 boxes contain fewer than 20 swings. One box recorded zero swings.

With this range of box sample sizes due to the varying spatial density of the data, the heat map at any resolution will contain boxes of exceedingly small sample sizes (high variance), and boxes of unnecessarily large sample size (unnecessarily low variance). Figure 1.4 shows six different resolutions for Peralta's data. We started with one box, and subdivided each box into four at each iteration. This simple resolution increase algorithm plainly illustrates the resolution selection challenge, and provides a foundation for our innovation in the next section. 
        \begin{figure}[H]
      	\centering
      	\includegraphics[scale=.35]{Images/Chapter_VarRes.png} 
      	\caption{These {\bf SIX (?)} heat maps show the same data for Johnny Peralta, at increasing resolutions. The maps range from obviously too coarse to perhaps excessively fine. Notice how dramatically the image changes as the resolution increases. Which resolution yields the highest quality heat map?}
      	\end{figure} 
Which of these five resolutions best balances spatially precise estimates of $p$ with acceptable box sample sizes? The viewer interested in the center of the strike zone might prefer the heat map six (NEED LABELS), as the box sample sizes support such spatially specific estimates. However, the boxes closer to the edges of the strike zone then contain higher variance, less reliable estimates due to prohibitively small sample sizes. 

All six resolution options include trade-offs. We propose a new heat map approach the eliminates trade-offs. The solution combines resolutions according to the data's varying spatial density.

\section{Spatially Varying Resolution} % ==========

Consider again the heat map in Figure 1.2. Recall box 22 contains 22 swings, a sample size where subdividing would yield sample sizes uselessly small, and thus estimate variances prohibitively high. In contrast, box 1694 would support estimates that are more spatially accurate without $\text{Var}(\hat{p}_{b})$ increasing beyond acceptable levels. We propose defining a stopping rule and a subdividing method, and subdividing individual boxes further accordingly. 

To illustrate, recall the $16 \times 16$ map in Figure 1.2. With this map as a starting point, we subdivide--into four smaller, equally sized boxes--all boxes where $n_{b} > 200$. Figure 1.5 shows this procedure. Moving through all boxes of the map to the left, subdividing when $n_{b} > 200$, yields the heat map to the right.
        \begin{figure}[H]
      	\centering
      	\includegraphics[scale=.3]{Images/Chapter4x4.pdf} 
      	\includegraphics[scale=.3]{Images/Chapter8x8_200.pdf} 
      	\caption{These heat maps show the empirical hitting success probability of Johnny Peralta, for pitches passing through the space represented by each square region of the hitting zone.  Each box maps $\hat{p}_{b}$ to a color, with box sample sizes, $n_{b}$, printed on box centers. All boxes with a sample size greater than 200 in the heat map on the left, have been subdivided in the heat map on the right. Box 22, and others like it, remain intact because further subdivision yields uninformatively low sample sizes.}
      	\end{figure} 
Notice all corner boxes remain intact. This indicates Peralta less often swings at pitches in these locations, and/or less often sees pitches in these locations. EXPLAIN WHY


The boxes toward the middle of the map generally contain more observations, and higher success probabilities $\hat{p}_{b}$. This indicates Peralta indeed hits pitches in the center of the hitting zone better than pitches toward the edges of the hitting zone. Also, Peralta sees and/or swings at more pitches toward the middle of the hitting zone. From the pitching side, pitchers throw through the middle of the hitting zone more frequently for a few reasons. First, aiming for the midde of the strike yields the highest probability of a strike; missing his target by up to approximately ten inches in any direction still yields a strike. On the other hand, when aiming away from the middle of the strike zone some pitches that miss their target will still {\it end up} in the middle of the strike zone. From the hitter's side, other things equal, these are the easiest pitches to hit. Peralta will swing at these pitches almost every chance he gets.

Sixteen boxes still have a sample size greater than 200, and 11 still have a sample size greater than 300. We iterate again, and further subdivide 16 boxes where $n_{b} > 200$.
        \begin{figure}[H]
      	\centering
      	
      	\includegraphics[scale=.25]{Images/Chapter4x4.pdf}
      	\includegraphics[scale=.25]{Images/Chapter8x8_200.pdf} 
      	\includegraphics[scale=.25]{Images/Chapter16x16_200.pdf} 
      	\caption{These heat maps convey the empirical batting average of batter 425509, Johnny Peralta, in each boxed region of the hitting zone. Each box maps $\hat{p}_{b}$ to a color. The number printed on each box represents the number of pitches the hitter swung at that passed through that box. All boxes with a sample size greater than 200 in the heat map on the left, have been subdivided in the heat map in the middle. All boxes with a sample size greater than 200 in the heat map in the middle, have been subdivided in the heat map on the right.}
      	\end{figure}
In Figure 6, the middle heat map has 16 boxes with $n_{b} > 200$. In the heat map on the right these 16 boxes have been subdivided into four boxes each. After this iteration, the heat map on the far right consists of 97 boxes, with a mean box sample size of 94.57, and median of 94. The minimum box sample size is 9, and the maximum is 189. The first quartile box sample size is 63, and the third quartile is 125. Regions with a higher density of pitch-swings necessarily have smaller boxes, which acts to convey additional information to the reader, compared to a heat map on a uniform grid. Note that the stopping rule and subdivision algorithm can be defined by the map's creator, offering flexibility to create the heat map structure that suits the data. 

Figure 8 gives the full sequence of heat maps that result from applying the stopping rule $n_{b} < 100$, starting with a single box.
        \begin{figure}[H]
      	\centering
      	\includegraphics[scale=.2]{Images/Chapter1x1.pdf}
      	\includegraphics[scale=.2]{Images/Chapter2x2.pdf}
      	\includegraphics[scale=.2]{Images/Chapter4x4.pdf}
      	\includegraphics[scale=.2]{Images/Chapter8x8_200.pdf} 
      	\includegraphics[scale=.2]{Images/Chapter16x16_200.pdf} 
      	\caption{These heat maps convey the empirical batting average of batter 425509, Johnny Peralta, in each boxed region of the hitting zone. Each box maps $\hat{p}_{b}$ to a color. The number printed on each box represents the number of pitches the hitter swung at that passed through that box. All boxes with a sample size greater than 200 in each heat map have been subdivided in the subsequent heat map.}
      	\end{figure}

To demonstrate the flexibility, consider a different stopping rule, $n_{b} < 100$. Figure 8 gives the sequence of heat maps that result from applying this stopping rule, with the same subdividing algorithm (***need to delineate what this algorithm is). 
        \begin{figure}[H]
      	\centering
      	\includegraphics[scale=.2]{Images/Chapter1x1.pdf}
      	\includegraphics[scale=.2]{Images/Chapter2x2.pdf}
      	\includegraphics[scale=.2]{Images/Chapter4x4.pdf}
      	\includegraphics[scale=.2]{Images/Chapter8x8_100.pdf}
      	\includegraphics[scale=.2]{Images/Chapter16x16_100.pdf}
      	\includegraphics[scale=.2]{Images/Chapter32x32_100.pdf}
      	\caption{These heat maps convey the empirical batting average of batter 425509, Johnny Peralta, in each boxed region of the hitting zone. Each box maps $\hat{p}_{b}$ to a color. The number printed on each box represents the number of pitches the hitter swung at that passed through that box. All boxes with a sample size greater than 100 in each heat map have been subdivided in the subsequent heat map.}
\end{figure} 	
Compare this sequence to Figure 7, where the stopping rule was $n_{b} < 100$. The top row of heat maps in Figure 7 and Figure 8 are identical, but notice in the four by four heat map that $100 < n_{(2,1)} < 200, \text{ and } 100 < n_{(1,4)} < 200$. This implies one stopping rule applies, but the other does not.
        \begin{figure}[H]
      	\centering      
      	\includegraphics[scale=.2]{Images/Chapter4x4.pdf}
      	\includegraphics[scale=.2]{Images/Chapter8x8_200.pdf}
      	\includegraphics[scale=.2]{Images/Chapter8x8_100.pdf}
      	\caption{...(these images, and others, need labels: (A) (B) (C) etc)}
\end{figure} 
For this reason the bottom left heat maps in Figures 7 and 8, shown in Figure 9, differ in the number of boxes of each size, and the total number of boxes. This divergence continues at the next iteration, where the stopping rule $n_{b} < 100$ requires 28 box subdivisions in Figure 8, map three; and $n_{b} < 200$ gives 16 box subdivisions in Figure 7, map three.

% *Alix: ``I wonder if there's any literature on the physics from a hitter's perspective in terms of how small a difference in location is even detectable.

% *Alix: ``Great start Chris. --> need to add more about how one would now interpret the ``best'' empirical heat map.

% \subsection{Appendix: VarResHM, An R Package}

% 
% \bibliography{Baseball}
% 
% \end{document}