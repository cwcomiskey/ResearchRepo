\section{Introduction} \label{intro}

\subsection{``A picture is worth a thousand words.''} 

Statisticians rely on graphical displays---pictures, essentially---to communicate information, analysis and results; but also as a diagnostic tool for model validation. As technology generates more data, and statistical analysis becomes more prevalent, graphical displays become more important; therefore, our graphic-making abilities must continue to evolve. The R data visualization package {\bf ggplot2} \citep{Wickham2009} was built specifically because of the importance of graphical displays, and the continued need for graphical innovations. Less than ten years old, {\bf ggplot2} is the most downloaded R package \citep{rdoc}.

In this chapter we focus on one type of graphical display: heat maps. Existing heat map displays consist of uniformly sized grid boxes across the domain. This means that a single resolution must suffice even if data concentration varies through the domain. To address this limitation, we introduce a variable-resolution (VR) heat map.

The remainder of Section \ref{intro} gives a bit of history, defines notation for Chapter 2, and introduces empirical hitting zone heat maps. In Section \ref{THM} we discuss traditional heat maps, and examine the problem of resolution selection. Next, in Section \ref{VRHM} we introduce VR heat maps as a solution to the resolution selection problem. In Section \ref{ETI} we show an example of VR heat maps with tornado data, and then discuss future research extensions in Section \ref{fr}. The last section contains a vignette for our VR heat map implementation package {\bf varyres}, available on GitHub (https://github.com/cwcomiskey).

\subsection{Heat Map History} % ============================

The first published heat map appears in 1873, when Toussaint Loua included one in his 1873 book of statistics about Paris. He shaded the squares by hand to represent levels of social characteristics across 20 Parisian districts \citep{Friendly2009}. In Figure \ref{fig:toua} we show Toua's map. 
  \begin{figure}[H]
	\centering
	\includegraphics[scale=.4]{Images/loua1873.jpg} 
  \caption{The earliest published heat map, published by Toussaint Loua in 1873. The graphic uses light to dark shading to convey the levels of social characteristics of 20 districts in Paris. Loua made graphic by hand, and it summarized 40 maps of Paris \citep{Friendly2009}.}
  \label{fig:toua}
	\end{figure}
Toua used this graphic to summarize information from 40 separate maps of Paris. 

Heat maps remain popular today, in part, because they are an incredible tool for summarizing information. In addition, they are especially effective with spatial data.

\subsection{Bernoulli Swings}

Baseball boils down to a series of contests between a hitter and a pitcher, in which the pitcher throws the ball and the hitter has to choose, to borrow from Shakespeare, to swing or not to swing. In our research we treat each \underline{swing} as a Bernoulli trial, and evaluate the success or failure of a swing independently from the count in the at bat. This differs from the norm; all other known research includes only the pitches that end an at bat \citep{Cross2015}, \citep{Baumer2010}, \citep{Fast2011}. As a result, these studies exclude swinging strikes that do not end at bats and foul balls; but include non-swinging strike three pitches.\footnote{Please see the appendix for the details and definitions of ball, strike, count, etc.} We consider the latter event a mistake of hitter decision making, not a failed swing attempt. 

Based on this interpretation, we define success as trials where the hitter gets a hit; and failure as trials where the hitter swings and misses, puts the ball in play for an out, or hits a foul ball.\footnote{We make this simplifying assumption---that a foul ball equals failure---even though some hitters are known to ``waste'' pitches with two strikes (i.e. Wade Boggs and Richie Ashburn), by intentionally hitting foul balls when they feel unable to do better with the pitch. }. In terms of the PITCHf/x\textsuperscript{\textregistered} data, this equates to success when variable ``\verb|des|'' (short for description)  equals ``\verb|in play, no out|''; and failure when ``\verb|des|'' equals ``\verb|Foul|'', ``\verb|Foul (Runner Going)|'', ``\verb|Foul Tip|'', ``\verb|In play out(s)|'', ``\verb|Swinging Strike|'', or ``\verb|Swinging Strike (Blocked)|''. Next we define notation, and explain the structure and interpretation of an empirical baseball hitting zone heat map.

\subsection{Empirical Hitting Zone Heat Maps} \label{EHZHM}

Empirical baseball hitting zone heat maps cover the two-dimensional, vertical face of the hitting zone with a grid, containing empirical success probabilities  in each grid box.  We start with PITCHf/x\textsuperscript{\textregistered} data on 1,932 right-handed hitters These hitters took 1,582,581 swings between 2008 and 2015. We denote these data as follows. Let
\bdm
S_i = \left\{\begin{array}{ll} 1; & \mbox{swing success} \\
					 0; & \mbox{swing failure} \\ \end{array} \right.
\edm
with $\pi_i = Pr(S_i = 1)$ for the $i = 1,2,\ldots, 1582581 = N$ swings.

Currently, we make the simplifying assumption that the $S_i$ are mutually independent for all $i$. This is a major, unrealistic assumption. An ideal interpretation would include correlation within at bats, games, seasons, pitch types, pitchers, etc. One consequence of this assumption will be artificially low estimation standard errors. However, for now we make this expedient assumption and proceed.

Let the $(x_i,y_i)$ pair denote the Cartesian coordinates of the location of the pitch, in the vertical face of the hitting zone, corresponding to the swing $i$. 

For the purpose of constructing a heat map, we consider the vertical face of the hitting zone to be partitioned into some number, $B$, of grid boxes depending on the resolution of the heat map. We let $G_1,G_2,\ldots,G_B$ index these grid boxes that partition the hitting zone. Now let
\bdm
N_b = \sum_{i=1}^N I_{(x_i,y_i) \in G_b},
\edm
where $I_{(x_i,y_i) \in G_b}$ is an indicator function that takes the value one if the $(x_i,y_i)$ coordinates are contained in $G_b$, and zero otherwise, for $b=1,2,\ldots,B$. We then calculate an empirical success probability {\em for each grid box}. Define $\pi_b = \mbox{Pr(Swing success in }G_b)$, and let
 \bdm
 p_b = \frac{1}{N_b} \sum_{i=1}^N S_i I_{(x_i,y_i) \in G_b}
 \edm
 for $b = 1,2,\ldots,B$.

In Figure \ref{fig:ms} we show a heat map of these empirical success probabilities using the PITCHf/x\textsuperscript{\textregistered} data and a resolution of $19 \times 33$ with $B = 627$ grid boxes. The graphic maps $p_{b}$ to a color on a spectrum, for pitches that passed through the space represented by that grid box.
  \begin{figure}[H]
	\centering
	\includegraphics[scale=.30]{Images/StrikeZone3.jpg} 
	\includegraphics[scale=.085]{Images/Mothership.jpg} 
  \caption{The yellow square in the image on the left is the strike zone, and coincides with the dashed blue line in the heat map on the right. The heat map grids the vertical face of the hitting zone with approximately 3/4 inch by 3/4 inch boxes. Each grid box color maps $p_{b}$---the empirical success probability of hitters swinging at pitches passing through that box---to a color.  The data consists of 1,932 right handed hitters, swinging at 1,582,581 pitches between 2008 and 2015.}
  \label{fig:ms}
	\end{figure} 
An important note is that the user determines the heat map resolution---in the case of Figure \ref{fig:ms} we used $19 \times 33$. In Figure \ref{fig:6res} we show an example of the same data at various resolutions. 

Another important thing to mention here is that each grid box in these single-resolution heat maps is given the same weight; put another way, each box is assumed to contain the same amount of information. This is not the case however, as there tend to be fewer pitches near the borders of the strike zone than near the middle. It is this problem that we tackle with VR heat maps.

\section{Traditional Heat Maps} \label{THM}

\subsection{Resolution} % ============== =============

By the time a viewer sees an empirical heat map, the analyst behind the graphic has already chosen a resolution. This important choice determines a uniform grid box size for the entire heat map, which considerably influences its quality and appearance considerably. To understand this influence, consider histograms, where bin width selection similarly affects histogram appearance and function. Small boxes may yield unreliable estimates at some locations, while larger boxes may fail to convey desired spatial specificity. 

Along these lines, note that heat maps conceal spatially varying data density. The map in Figure \ref{fig:ms} gives no indication of the number of observations in different regions of the hitting zone. In this way, empirical heat maps also implicitly conceal sample sizes and hence the variances of the estimates they display.  

With these points in mind, consider again the heat map in Figure \ref{fig:ms}. That heat map divides the hitting zone into relatively small boxes, because the data supports it; approximately 1.5 million swings by almost 2000 hitters. By ``supports it'' we mean that the small, spatially specific boxes at this resolution retain sample sizes large enough to supply reasonable estimates of $\pi_{b}$. ``Reasonable,'' of course, depends on context and objectives. For example, a pitching coach might request estimates within 0.005 points of the true batting average with confidence 0.95. This requires a sample size of at least 36 when $\pi_{b} = 0.10$. Note that for Bernoulli random variables $\text{Var}(p_{b})$ depends on $\pi_{b}$. This creates counterintuitive behavior around the margins of the hitting zone, where $\text{Var}(p_{b})$ remains very small despite very small box sample sizes. See \cite{Dixon2005} for more on this curious phenomenon.

On the other hand, data for individual hitters varies dramatically in size. In our database, individual hitters have numbers of swings ranging from a single swing to over 10,000 swings. At such varying scales, resolution selection becomes more complicated because non-uniformly dispersed data implies that different regions may support very different resolutions. This is important because, as stated before, the choice of resolution sometimes dramatically affects heat map appearance, but also its usefulness. For example, coarse resolution in regions of interest means that information content may not be optimized. 

This important resolution decision usually depends on the size and nature of the dataset, and its dispersion through the spatial domain. In the next section we explore resolution selection in detail, along with its inherent compromises.

\subsection{Resolution Selection}

To illustrate some of the issues around resolution selection we use batter 425509, a veteran player named Jhonny Peralta. The data over the 2008 - 2015 time frame include 9,177 Peralta swings, and these swings yield, when B = 16 with boxes of equal size, the heat map in Figure \ref{fig:4x4}.
Each box maps $p_{b}$ to a color, and we have printed box sample sizes at the center of each box. We will use the numbering of the $4 \times 4$ grid on the right to reference the corresponding heat map boxes. 
        \begin{figure}[H]
      	\centering
      	\includegraphics[scale=.08]{Images/Chapter4x4.jpg} 
      	\includegraphics[scale=.08]{Images/indexes.jpg}
      	\caption{The $4 \times 4$ heat map (LHS) shows the empirical batting average of Jhonny Peralta, for pitches passing through the space represented by each of 16 square regions of the hitting zone. Each box maps $p_{b}$ to a color, with box sample sizes printed on box centers. The map on the right provides indexes to identify boxes.}
      	\label{fig:4x4}
      	\end{figure} 
Notice the $4 \times 4$ resolution gives box 13 a sample size of 22, so that further subdivision of that box might yield trivially small sample sizes. On the other hand, central boxes 6, 7, 10, and 11---all with sample sizes above 1200---can and we believe should contribute more location-specific estimates. Therefore, the central box sample sizes motivate finer resolution, even though box 13 does not support it. Keeping this trade-off in mind, we increase resolution by dividing each box into four sub-boxes with equal area. In Figure \ref{fig:8x8} we show the $8 \times 8$ resolution result.
        \begin{figure}[H]
      	\centering
      	\includegraphics[scale=.1]{Images/Chapter8x8.jpg} 
      	\caption{This eight by eight heat map shows the empirical hitting success probability of Jhonny Peralta, for pitches passing through the space represented by each of 64 square, equal area regions of the hitting zone. Each box maps $p_{b}$ to a color, with box sample sizes, $N_{b}$, printed on box centers. A transparent box indicates no pitches passed through that box. Notice that this resolution imparts additional information in the center of the hitting zone, but some box sample sizes near the margins have dropped to very low values.}
      	\label{fig:8x8}
      	\end{figure} 
The centermost 16 boxes will yield low variance estimates of $\pi_{b}$, with standard errors of $p_{b}$ ranging from 0.013 to 0.019.\footnote{Recall the independence assumption implies these standard errors are artificially low; but they are useful for comparisons to boxes with larger sample sizes.}  Overall, 24 boxes contain 150 swings or more; and 15 boxes contain more than 250 swings. We believe that subdividing these boxes even more would yield additional information. On the other hand, many boxes, especially edge boxes, now have problematically small sample sizes. For example, 19 of 27 remaining edge boxes have $p_{b} = 0$. The remaining eight boxes have standard errors ranging from 0.026 to 0.088. Overall, 29 boxes contain fewer than 50 swings, and 17 boxes contain fewer than 20 swings. One box contains zero swings.

Figure \ref{fig:6res} shows Peralta's data---the same data---at a sequence of six resolutions: $2^{\text{r}}\times 2^{\text{r}}$, for $r = 0, \dots, 5$.
        \begin{figure}[H]
      	\centering
      	\includegraphics[scale=.055]{Images/Chapter_VarRes.jpg} 
      	\caption{These six heat maps show the same data for Jhonny Peralta, at increasing resolutions: $2^{\text{r}}\times 2^{\text{r}}$, for $r = 0, \dots, 5$. The maps range from too coarse to too fine. Notice how dramatically the image changes as the resolution increases. Which resolution yields the highest quality heat map?}
      	\label{fig:6res}
      	\end{figure} 
We started with one box, and then subdivided each box into four smaller boxes of equal area to create the next map. Which of these six resolutions best balances spatially precise estimates of $p$ with acceptable box sample sizes? The user interested in the center of the strike zone might prefer map (E). However, the boxes closer to the edges of the strike zone then contain prohibitively small sample sizes, yielding higher variance estimates. 

The takeaway from Figure \ref{fig:6res}: all six resolutions involve trade-offs. We propose a new heat map approach that eliminates trade-offs. The solution combines multiple resolutions into one map, according to the varying abundance of data through the domain.

\section{Variable-Resolution Heat Maps} \label{VRHM}

\subsection{The Varying Resolution Solution}

Consider again the Peralta $4 \times 4$ heat map in Figure \ref{fig:4x4}. Recall box 13 contains 22 swings, a sample size where subdividing further yields trivially small sample sizes. By contrast, subdividing box 6 may result in estimates that are more spatially accurate without $\text{Var}(p_{b})$ increasing beyond acceptable levels. To resolve this problem, we propose deciding resolution increases algorithmically, box by box, until a stopping rule is satisfied; we call it the variable-resolution (VR) algorithm. The map producer chooses a stopping rule, allowing flexibility to create a heat map that is appropriate for the available data. 

To demonstrate one iteration in the VR algorithm, let the stopping rule be a maximum box sample size of 200. By this we mean that we will stop subdividing a box when $N_{b} \leq 200$. Recall the $4 \times 4$ map in Figure \ref{fig:4x4} (also Figure \ref{fig:4x4and8x8}, LHS). On this map we divide all boxes where $N_{b} > 200$, into four smaller boxes of equal area. In Figure \ref{fig:4x4and8x8} we show the map at $4 \times 4$ resolution (LHS) before subdividing, and with variable resolution (RHS) after subdividing.
        \begin{figure}[H]
      	\centering
      	\includegraphics[scale=.08]{Images/Chapter4x4.jpg} 
      	\includegraphics[scale=.08]{Images/Chapter8x8_200.jpg} 
      	\caption{One iteration in the variable-resolution algorithm. The algorithm subdivides all boxes with a sample size (printed on box) greater than 200. The single iteration shown here yields the map on the right, from the map on the left.}
      	\label{fig:4x4and8x8}
      	\end{figure} 
Boxes with fewer than 200 observations, such as box 13, remain intact because further subdivision yields sample sizes deemed too small. Sixteen boxes still contain a sample size greater than 200, and 11 still have a sample size greater than 300. In the next iteration of our algorithm, which we show in Figure \ref{fig:8x8to16x16}, we subdivide all remaining boxes with more than 200 observations. 
        \begin{figure}[H]
      	\centering
      	\label{fig:8x8to16x16}
      	\includegraphics[scale=.08]{Images/Chapter8x8_200.jpg} 
      	\includegraphics[scale=.08]{Images/Chapter16x16_200.jpg} 
      	\caption{Two iterations in the variable-resolution algorithm. The algorithm subdivides all boxes with a sample size (printed on box) greater than 200. This iteration yields the map on the right, from the map in the middle.}
      	\end{figure}
Note the inverse correspondence between box size and observation abundance; smaller grid boxes correspond to greater observation abundance.

In Figure \ref{fig:allvr} we show the VR heat map at every iteration of our algorithm using the stopping rule $N_{b} < 200$.
        \begin{figure}[H]
      	\centering
      	\includegraphics[scale=0.05]{Images/6_200.jpg}
      	\caption{Variable-resolution heat map sequence. Starting from the top left, the algorithm subdivides all boxes with a sample size (printed on box) greater than 200. The maps convey Jhonny Peralta's empirical success probability by mapping $p_{b}$ to a color.}
      	\label{fig:allvr}
      	\end{figure}
      	
Next we provide pseudo-code for our VR algorithm.
      	
\subsection{Variable-Resolution Algorithm}

The VR algorithm requires a user make two primary decisions: subdivision method and stopping rule type and threshold. We subdivide into four boxes of equal area, but this could be modified to suit different scenarios. For example, a user could subdivide into six boxes of equal area by making two horizontal subdivisions and one vertical; or even subdivide into boxes of unequal area.

For a stopping rule, we use $N_{b} < 200$. A user could modify this by type or threshold. For example, a user could modify the stopping rule type by using $\text{SE}(p_{b}) > \sigma_{0}$; and modify the stopping rule threshold by using $\text{SE}(p_{b}) > \sigma_{1}$, where $\sigma_{1} > \sigma_{0}$.

Using a stopping rule based on a box sample size threshold, our algorithm in commented pseudo-code follows.
\begin{verbatim}
function(dataset, 
         fun = mean, # function to apply to box data
         cutoff,     # stopping rule threshold
         max ){      # maximum number of iterations
         
Create HISTORY_CONTAINER # to store every iteration result
ITERATION <- 0              
Create CURRENT_CONTAINER and store: box centers, 
  statistic of interest, observations per box, 
  box heights/widths, box vertical/horizontal upper/lower 
  bounds
Add first CURRENT CONTAINER to HISTORY CONTAINER as first
  element
Add CURRENT_CONTAINER to HISTORY_CONTAINER as its first
  element

WHILE(exists N_b > cutoff and ITERATION < max) {
    ITERATION <- ITERATION + 1    
    Create ITERATION CONTAINER
            
            FOR(All BOX_b in CURRENT CONTAINER){
            
                IF(BOX_b count > cutoff){ 
                    Retrieve BOX_b data, create BOX_b
                      CONTAINER
                    SUB-BOXES_CURRENT_CONTAINER: 
                      subdivide BOX_b, calculate BOX_b 
                      sub-box information
                    Add SUB-BOXES_CURRENT_CONTAINER to
                      ITERATION CONTAINER  
                  } # END "IF" statement
                        
              }  # END "FOR" loop
              
    Update CURRENT_CONTAINER: remove obsolete boxes,
      and add ITERATION CONTAINER boxes
    # Add CURRENT CONTAINER to HISTORY CONTAINER 
      as element "ITERATION + 1"
            
  }  # END "WHILE" loop
  }

\end{verbatim}
      	
\subsection{Data Density Information}

In VR heat maps, regions with more observations have smaller boxes, and thus more spatially exact estimates. In this way the size of the box conveys sample size information, which existing heat maps typically do not show. This new feature derives from the fact that box subdivisions persist until box sample sizes drop below 200 (under this stopping rule). In Figure \ref{fig:density} we show the correspondence between observation abundance and box size, by comparing a spatial scatterplot of observation (pitch) locations to a VR heat map. 
        \begin{figure}[H]
      	\centering
      	\includegraphics[scale=.07]{Images/density.jpg}
      	\includegraphics[scale=.07]{Images/density_comp.jpg} 
      	\caption{Variable-resolution (VR) heat maps convey the relative abundance of observations throughout the domain. Comparing Jhonny Peralta's swing location scatterplot and VR heat map shows the correspondence between sample size and resolution; finer resolution (smaller boxes) corresponds to relatively more observations, whereas more coarse resolution (larger boxes) indicates relatively fewer observations. Existing heat maps do not show sample size information.}
      	\label{fig:density}
      	\end{figure}
In Figure \ref{fig:density} notice how smaller boxes indicate higher data density, while larger boxes indicate lower density. This contrasts favorably to traditional heat maps, where uniform resolution conceals the number of observations behind each grid box. VR heat maps convey valuable, previously omitted, data density information to the viewer. 
      	
\subsection{Example: Alternate Stopping Rule} % ===================
      	
In this section we apply the VR algorithm to the same data, but with stopping rule $N_{b} < 100$ instead of $N_{b} < 200$. A comparison with the previous VR maps shows how the algorithm's iterations and outcome change with a different stopping rule. Figure \ref{fig:altsr} shows heat maps for all six VR iterations.
        \begin{figure}[H]
      	\centering
      	\includegraphics[scale=0.05]{Images/6_100.jpg}
      	\caption{A variable-resolution heat map sequence. Starting from the top left and moving across, the algorithm subdivides all boxes with a sample size (printed on box) greater than 100. The maps convey Jhonny Peralta's empirical success probability by mapping $p_{b}$ to a color.}
      	\label{fig:altsr}
\end{figure} 	
Compare Figure \ref{fig:altsr} for stopping rule $N_{b} < 100$, to Figure \ref{fig:allvr} for stopping rule $N_{b} < 200$: the first three maps in each sequence match exactly. However, notice the resolution toward the center of the last heat map increased beyond what we saw before with $N_{b} < 200$ maps. For baseball strike zone data this resolution increase may be inappropriate if hitters cannot differentiate between adjacent locations, but the cutoff component of the algorithm offers the flexibility to make that assessment and choose accordingly. Also, notice there are boxes in the $4 \times 4$ heat map that have sample sizes {\it between} the two stopping rules. For these boxes one stopping rule prevents further subdivision, while the other compels it, which leads to different final heat maps. In Figure \ref{fig:vrcomp} we show the subsequent map at this iteration for each stopping rule, with $N_{b} < 200$ on the left, and $N_{b} < 100$ on the right.
        \begin{figure}[H]
      	\centering      
      	\includegraphics[scale=.08]{Images/Chapter8x8_200.jpg}
      	\includegraphics[scale=.08]{Images/Chapter8x8_100.jpg}
      	\caption{Variable-resolution heat maps diverge when box totals fall between sample size stopping rules. Boxes 158 and 102 remain for stopping rule $N_{b} < 200$ (left), but further subdivide for stopping rule $N_{b} < 100$ (right).}
      	\label{fig:vrcomp}
\end{figure} 
The ``$N_{b} < 100$'' map now offers more spatial specificity at the former locations of Box 102 and Box 158. The heat maps also now differ in the number of boxes of each size, and the total number of boxes.  The differences increase at the next iteration, where stopping rule $N_{b} < 100$ produces 28 box subdivisions (Figure \ref{fig:altsr}); and $N_{b} < 200$ produces 16 box subdivisions (Figure \ref{fig:allvr}). Figure \ref{fig:vrcomp2} shows two corresponding iterations for these two stopping rules with $N_{b} < 100$ in the top row, and $N_{b} < 200$ in the bottom row. For both stopping rules, notice how corner boxes tend to remain intact.
        \begin{figure}[H]
      	\centering      
      	\includegraphics[scale=.05]{Images/Chapter4x4.jpg}
      	\includegraphics[scale=.05]{Images/Chapter8x8_200.jpg}
      	\includegraphics[scale=.05]{Images/Chapter16x16_200.jpg}
      	\includegraphics[scale=.05]{Images/Chapter4x4.jpg}
      	\includegraphics[scale=.05]{Images/Chapter8x8_100.jpg}
      	\includegraphics[scale=.05]{Images/Chapter16x16_100.jpg}
      	\caption{Variable-resolution heat maps diverge when box totals fall between sample size stopping rules. The top row of maps subdivides according to stopping rule $N_{b} < 200$; the bottom row further subdivides for stopping rule $N_{b} < 100$. The first three iterations produce identical maps, but then the sequences diverge as shown here.}
      	\label{fig:vrcomp2}
\end{figure}
Users should choose a stopping rule based on situation specific factors. One factor is the sample sizes required to provide estimates with acceptable standard errors. A second factor is the desired spatial specificity of estimates, considering the practical relevance. For example, hitters may not be able to distinguish between pitch locations 1/2 inch apart, so we would not choose a stopping rule that yields estimates spatially specific to 1/2 inch.

\subsection{Interpretation: Hitter vs. Pitcher} % ==================

All-Star first baseman Keith Hernandez said the ``battle of wits... between the pitcher and hitter is baseball. Everything else is secondary.'' At the most basic level, the pitcher wants to get outs and avoid baserunners; and the hitter wants to avoid outs and get on base. There are two sides to every pitch's location: the pitcher's decision to throw there, and the hitter's decision to swing. In this section we look at two primary factors that influence these decisions in the battle of wits: game theoretic strategy, and pitch location execution error. We then use them to interpret a Peralta VR heat map.

\subsubsection*{Background}

Game theoretic strategy concerns the pitcher's knowledge of the hitter's strengths and weaknesses, and the hitter's reciprocal knowledge. For example, a pitcher avoids throwing pitches to the locations a hitter succeeds with the highest probability. In turn, the hitter would like to avoid swinging at pitches in locations he succeeds with relatively low probability.

Pitch location execution error refers to the fact that pitchers aim for the catcher's carefully positioned glove, the mutually decided target, but only rarely hit it exactly. More commonly the catcher moves his glove some distance to catch the pitch.  Analysis of this distance is impossible, because PITCHf/x\textsuperscript{\textregistered} data does not record the catcher's initial glove location. Therefore, without data to suggest otherwise, based on this author's own experience playing and watching baseball, for this discussion we assume a symmetric, ``bell-shaped'' distribution for each pitch's horizontal and vertical location error.

\subsubsection*{Interpretation}
Now we use game theory and pitch location error to interpret aspects of a Peralta VR heat map. Figure \ref{fig:interp} shows Peralta's VR heat map for stopping rule $N_{b} < 200$.
        \begin{figure}[H]
      	\centering      
      	\includegraphics[scale=.1]{Images/Peralta_var-res2.jpg}
      	\caption{This variable-resolution heat map for Jhonny Peralta gives spatial, empirical success probabilities; each box maps $p_{b}$ to a color. In addition, the map conveys data density information; box size corresponds to the observation density, because subdivision persisted until box sample sizes dropped below 200. The dotted line outlines the strike zone.}
      	\label{fig:interp}
        \end{figure}
First, notice the grid boxes inside the dotted blue strike zone box tend to be smaller; this indicates Peralta generally saw and swung at far more pitches in the strike zone than out.\footnote{Please see the Appendix for the details and definitions of ball, strike, count, etc.} Pitchers throw pitches in the strike zone because each strike moves the at bat one step closer to an out; and each ball moves the at bat one step closer to a walk, where hitter proceeds to first base. The same logic, with reversed incentives, explains why Peralta swings at so many pitches in the strike zone. 

Second, notice the larger boxes in the lower left, upper left, and upper right of the strike zone. The larger boxes imply Peralta saw and/or swung at fewer pitches in these locations. To explain, consider a pitch aimed for the bottom left of the strike zone. Using the pitch location error distribution we assumed, aiming for this box yields a ball approximately 3/4 of the time, a result in the hitter's favor. From the hitter's perspective, these corner pitches are relatively difficult to hit, so Peralta presumably avoids swinging at them when possible. These two strategic considerations combined help explain low observation density in these locations.

Third, Peralta seems to have more success in the horizontal center of the strike zone, than at the top or bottom. This suggests Peralta handles horizontal variation better than vertical variation---an actionable insight. Since Peralta swings at pitches in the middle of the strike zone almost every chance he gets, why do pitchers not avoid those locations? They probably tried. However, assuming a symmetric, bell-shaped distribution for pitch location error, aiming for the middle of the strike zone offers the highest probability of a strike. In addition, some pitches aimed toward the margins of the strike zone miss the target and unintentionally pass through the middle of the strike zone. 

Finally, of the four corner locations just outside the strike zone, Peralta enjoyed the most success in the lower-left (yellow) box, with $p_{b} = 0.10$. However, the size of this box indicates a relatively small sample size; either Peralta astutely swings at these challenging pitches infrequently, or pitchers may throw there infrequently because of the relatively greater risk: they less commonly induce swings, and non-swings move the at bat one step closer to a successful outcome. 

VR heat maps also work well with other geostatistical data. In the next section we give an example of VR maps with tornado data.

\section{Example: Tornado Intensity} \label{ETI}

\subsection{Background}

The National Weather Service, a division of the National Oceanic and Atmospheric Administration (NOAA), maintains seven National Centers for Environmental Prediction. One of these, the Storm Prediction Center (SPC), collects tornado data, and provides it to the public at their website http://www.spc.noaa.gov/gis/svrgis/ \citep{NOAA}. In this section we use SPC spatial tornado data, collected between 1950 and 2017, to show VR heat maps in another area of application. 

Meteorologists rate tornado intensity on the Fujita-scale (F-Scale), based primarily on tornado damage. The dataset we use includes the longitude, latitude, and F-Scale rating for approximately 30,000 tornadoes in the last 67 years. Ordinarily these data would make resolution selection difficult, because tornado frequency varies greatly across the U.S. Low resolution appropriate for the Western U.S. would fail to convey available detail in the Midwest and West South Central division of the Southern U.S.\footnote{See \cite{regions} for region definitions.} High resolution appropriate for the latter regions would lead to sparsely populated grid boxes in the Western U.S. A VR heat map solves this challenge, and at the same time conveys the varying tornado frequency. 

\subsection{Variable-Resolution Map}

The VR heat map in Figure \ref{fig:tornado1}, created with stopping rule $N_{b} < 50$ shows the spatial average intensity, and relative frequency, of tornadoes.\footnote{We use stopping rule $N_{b} < 50$ without statistical justification, but rather for the aesthetics of the final heat map for the example.}
        \begin{figure}[H]
      	\centering      
      	\includegraphics[scale=.09]{Images/vr_tornados.jpg}
      	\caption{This variable-resolution heat map shows tornado F-Scale intensity and frequency across the U.S. Each box maps average intensity to a color. Smaller (larger) boxes correspond to higher (lower) observation density, because subdivision persisted until box sample sizes dropped below 50 tornadoes.}
      	\label{fig:tornado1}
        \end{figure}
Notice how the variable-resolution map conveys tornado frequency through grid box size, and seems to divide the country into three vertical regions. To the East of $-80^{\circ}$ West, the medium-sized boxes indicate with their size and color moderate tornado frequency and intensity. Between approximately $-80^{\circ}$ West and $-100^{\circ}$ West, tornado intensity and frequency increases, indicated by the color and area of these smaller boxes. Finally, moving further West, these ligher, bigger boxes indicate less frequent and less intense tornados than either of the previous two longitudinal ranges. The map performs well in these ways, but produces at least one peculiarity.

Notice how some grid boxes remain unnecessarily large, and awkwardly shaped relative to the U.S. borders. For example, a large horizontal rectangle overlaps Maine, in the North-eastern most corner of the U.S. This phenomenon occurs when an early iteration subdivision leaves a protruding quadrant with at least one observation, but fewer than the cutoff. The VR algorithm iterations in Figure \ref{fig:tornado2} show how the VR map evolves to its final form, but retains protruding boxes.
        \begin{figure}[H]
      	\centering      
      	\includegraphics[scale=.06]{Images/tornado_maps.jpg}
      	\caption{Variable-resolution (VR) algorithm iterations. The box colors represent average tornado F-Scale intensity, and the sequence shows the VR heat map evolution. Subdivision persists until box sample sizes drop below 50 (in this case) tornadoes in each box.}
      	\label{fig:tornado2}
        \end{figure}
The protruding box pattern occurred with boxes that protrude from, and partially overlap New Jersey, Southern Florida, Southern Texas, and Southern California. However, the size of these odd boxes does still indicate very low observation density, alerting the audience of the circumstance. Modifications to the algorithm---subdividing edge boxes with this quality---would eliminate this quirk.

\section{Future Research} \label{fr}

Future research could enhance the VR algorithm by adding stopping rule types or subdivision methods. For example, straightforward subdivision into boxes with equal area sometimes yields boxes with very unequal sample sizes. One alternative to prevent this is to subdivide {\it according} to sample size; horizontal and vertical divisions calculated to yield grid boxes of approximately equal numbers of observations. The resulting boxes would have similar sample sizes rather than equal area.

Also, future research should analyze PITCHf/x\textsuperscript{\textregistered} hitting zone data without assuming independence. Sparse data will make this avenue challenging; subsetting the observations according to expected categories of within-group correlation---such as at bats, games, seasons, pitch types, pitchers, etc.---will yield small sample sizes to analyze.

\section{Conclusion}

In this chapter we introduced the framework we use for analyzing PITCHf/x\textsuperscript{\textregistered} hitting zone data, and creating the empirical heat maps. After discussing traditional heat maps we analyzed resolution selection, and showed that with existing heat maps this decision involves trade-offs. To address this issue, we introduced VR heat maps. We gave the VR algorithm in pseudo-code, showed a VR map of the same data with an alternate stopping rule, and then showed an example with tornado data. Last, we described two extensions for future research. 

Following this conclusion is a vignette for our VR heat map implementation package {\bf varyres}, available on GitHub (https://github.com/cwcomiskey). 

\section{varyres: An R Package}

\includepdf[pages=-, scale=0.9, pagecommand={}]{varyres.pdf}
